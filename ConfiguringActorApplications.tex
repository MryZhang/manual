\section{Configuring Actor Applications}
\label{system-config}

\lib configures applications at startup using an \lstinline^actor_system_config^ or a user-defined subclass of that type. The config objects allow users to add custom types, to load modules, and to fine-tune the behavior of loaded modules with command line options or configuration files~\see{system-config-options}.

The following code example is a minimal CAF application with a middleman~\see{middleman} but without any custom configuration options.

\begin{lstlisting}
void caf_main(actor_system& system) {
  // ...
}
CAF_MAIN(io::middleman)
\end{lstlisting}

The compiler expands this example code to the following.

\begin{lstlisting}
void caf_main(actor_system& system) {
  // ...
}
int main(int argc, char** argv) {
  return exec_main<io::middleman>(caf_main, argc, argv);
}
\end{lstlisting}

The function \lstinline^exec_main^ creates a config object, loads all modules requested in \lstinline^CAF_MAIN^ and then calls \lstinline^caf_main^. A minimal implementation for \lstinline^main^ performing all these steps manually is shown in the next example for the sake of completeness.

\begin{lstlisting}
int main(int argc, char** argv) {
  actor_system_config cfg;
  // read CLI options
  cfg.parse(argc, argv);
  // return immediately if a help text was printed
  if (cfg.cli_helptext_printed)
    return 0;
  // load modules
  cfg.load<io::middleman>();
  // create actor system and call caf_main
  actor_system system{cfg};
  caf_main(system);
}
\end{lstlisting}

However, setting up config objects by hand is usually not necessary. \lib automatically selects user-defined subclasses of \lstinline^actor_system_config^ if \lstinline^caf_main^ takes a second parameter by reference, as shown in the minimal example below.

\begin{lstlisting}
class my_config : public actor_system_config {
public:
  my_config() {
    // ...
  }
};

void caf_main(actor_system& system, const my_config& cfg) {
  // ...
}

CAF_MAIN()
\end{lstlisting}

Users can perform additional initialization, add custom program options, etc. simply by implementing a default constructor.

\subsection{Loading Modules}
\label{system-config-module}

The simplest way to load modules is to use the macro \lstinline^CAF_MAIN^ and to pass a list of all requested modules, as shown below.

\begin{lstlisting}
void caf_main(actor_system& system) {
  // ...
}
CAF_MAIN(mod1, mod2, ...)
\end{lstlisting}

Alternatively, users can load modules in user-defined config classes.

\begin{lstlisting}
class my_config : public actor_system_config {
public:
  my_config() {
    load<mod1>();
    load<mod2>();
    // ...
  }
};
\end{lstlisting}

The third option is to simply call \lstinline^x.load<mod1>()^ on a config object \emph{before} initializing an actor system with it.

\subsection{Command Line Options and INI Configuration Files}
\label{system-config-options}

\lib organizes program options in categories and parses CLI arguments as well as INI files. CLI arguments override values in the INI file which override hard-coded defaults. Users can add any number of custom program options by implementing a subtype of \lstinline^actor_system_config^. The example below adds three options to the ``global'' category.

\lstinputlisting[linerange={250-262}]{../examples/remoting/distributed_calculator.cpp}

The line \lstinline^opt_group{custom_options_, "global"}^ adds the ``global'' category to the config parser. The following calls to \lstinline^add^ then append individual options to the category. The first argument to \lstinline^add^ is the associated variable. The second argument is the name for the parameter, optionally suffixed with a comma-separated single-character short name. The short name is only considered for CLI parsing and allows users to abbreviate commonly used option names. The third and final argument to \lstinline^add^ is a help text.

The custom \lstinline^config^ class allows end users to set the port for the application to 42 with either \lstinline^--port=42^ (long name) or \lstinline^-p 42^ (short name). The long option name is prefixed by the category when using a different category than ``global''. For example, adding the port option to the category ``foo'' means end users have to type \lstinline^--foo.port=42^ when using the long name. Short names are unaffected by the category, but have to be unique.

Boolean options do not require arguments. The member variable \lstinline^server_mode^ is set to \lstinline^true^ if the command line contains either \lstinline^--server-mode^ or \lstinline^-s^.

\lib prefixes all of its default CLI options with \lstinline^caf#^, except for ``help'' (\lstinline^--help^, \lstinline^-h^, or \lstinline^-?^). The default name for the INI file is \lstinline^caf-application.ini^. Users can change the file name and path by passing \lstinline^--caf#config-file=<path>^ on the command line.

INI files are organized in categories. No value is allowed outside of a category (no implicit ``global'' category). CAF reads \lstinline^true^ and \lstinline^false^ as boolean, numbers as (signed) integers or \lstinline^double^, \lstinline^"^-enclosed characters as strings, and \lstinline^'^-enclosed characters as atoms~\see{atom}. The following example INI file lists all standard options in \lib and their default value. Note that some options such as \lstinline^scheduler.max-threads^ are usually detected at runtime and thus have no hard-coded default.

\clearpage
\lstinputlisting[language=ini]{../examples/caf-application.ini}

\clearpage
\subsection{Adding Custom Message Types}
\label{add-custom-message-type}

\lib requires serialization support for all of its message types \see{type-inspection}. However, \lib also needs a mapping of unique type names to user-defined types at runtime. This is required to deserialize arbitrary messages from the network.

As an introductory example, we (again) use the following POD type \lstinline^foo^.

\lstinputlisting[linerange={24-27}]{../examples/custom_type/custom_types_1.cpp}

To make \lstinline^foo^ serializable, we make it inspectable \see{type-inspection}:

\lstinputlisting[linerange={30-34}]{../examples/custom_type/custom_types_1.cpp}

Finally, we give \lstinline^foo^ a platform-neutral name and add it to the list of serializable types by using a custom config class.

\lstinputlisting[linerange={75-78,81-84}]{../examples/custom_type/custom_types_1.cpp}

\subsection{Adding Custom Error Types}

Adding a custom error type to the system is a convenience feature to allow improve the string representation. Error types can be added by implementing a render function and passing it to \lstinline^add_error_category^, as shown in~\sref{custom-error}.

\clearpage
\subsection{Adding Custom Actor Types \experimental}
\label{add-custom-actor-type}

Adding actor types to the configuration allows users to spawn actors by their name. In particular, this enables spawning of actors on a different node \see{remote-spawn}. For our example configuration, we consider the following simple \lstinline^calculator^ actor.

\lstinputlisting[linerange={33-39}]{../examples/remoting/remote_spawn.cpp}

Adding the calculator actor type to our config is achieved by calling \lstinline^add_actor_type<T>^. Note that adding an actor type in this way implicitly calls \lstinline^add_message_type<T>^ for typed actors \see{add-custom-message-type}. This makes our \lstinline^calculator^ actor type serializable and also enables remote nodes to spawn calculators anywhere in the distributed actor system (assuming all nodes use the same config).

\lstinputlisting[linerange={99-101,106-106,110-101}]{../examples/remoting/remote_spawn.cpp}

Our final example illustrates how to spawn a \lstinline^calculator^ locally by using its type name. Because the dynamic type name lookup can fail and the construction arguments passed as message can mismatch, this version of \lstinline^spawn^ returns \lstinline^expected<T>^.

\begin{lstlisting}
auto x = system.spawn<calculator>("calculator", make_message());
if (! x) {
  std::cerr << "*** unable to spawn calculator: "
            << system.render(x.error()) << std::endl;
  return;
}
calculator c = std::move(*x);
\end{lstlisting}

Adding dynamically typed actors to the config is achieved in the same way. When spawning a dynamically typed actor in this way, the template parameter is simply \lstinline^actor^. For example, spawning an actor "foo" which requires one string is created with \lstinline^system.spawn<actor>("foo", make_message("bar"))^.

Because constructor (or function) arguments for spawning the actor are stored in a \lstinline^message^, only actors with appropriate input types are allowed. For example, \lstinline^const char*^ arguments---or any other pointer type---are not allowed and must be replaced by \lstinline^std::string^.
